%
\documentclass[Proceedings]{ascelike}
%
% Feb. 14, 2013
%
% Some useful packages...
%
\usepackage{graphicx}
%\usepackage{subfigure}
\usepackage{amsmath}
%\usepackage{amsfonts}
%\usepackage{amssymb}
%\usepackage{amsbsy}
%\usepackage{times}
%
%
% Place hyperlinks within the pdf file (works only with pdflatex, not latex)
% \usepackage[colorlinks=true,citecolor=red,linkcolor=black]{hyperref}
%
%
% NOTE: Don't include the \NameTag{<your name>} if you have selected 
%       the NoPageNumbers option: this leads to an inconsistency and
%       a warning, and the NameTag is ignored.
%\NameTag{Kuhn, Feb. 14, 2013}
%
%
\begin{document}
%
% You will need to make the title all-caps
\title{Determinaci\'on del periodo de la orbita de una estrella binaria espectrosc\'opica.}
%
\author{
Nicolas Garavito-Camargo,%
%
% ---- The first of two styles for addresses: using footnotes and \thanks ----
\thanks{
Dept.\ de F\'isica.,
Universidad de los Andes, 
Cra 1 Nº 18A- 12\ Bogot\'a, Colombia. E-mail: jn.garavito57@uniandes.edu.co}
\ Benjamin Oostra\footnotemark[1]
%
% Adding a second author with the same affiliation (still using \thanks):
%  \\
 %Colleague,\footnotemark[1] Member, ASCE%
%
% Adding another author with a different affiliation.  I have found that 
% the \and command doesn't quite work, so just use "and", as in the following 
% \\
% and
% Younyee Kuhn%
% \thanks{Flourishing wife of same.},%
% \ Not a Member, ASCE
%
% ---- The second of two styles for addresses: below names, no footnotes ----
%
% For this style, don't use \thanks.  Instead, use superscripts and carriage
% returns ("\\").  It's not pretty, but neither is the new ASCE proceedings
% style.  Something like the following:
%
% Matthew R. Kuhn$^1$, Member, ASCE\\[1ex]%
%
% $^1$\parbox[t]{5.75in}{Dept.\ of Civil Engrg.,
% Donald P.\ Shiley School of Engrg., Univ.\ of Portland, 
% 5000 N.\ Willamette Blvd., Portland, OR  97203. kuhn@up.edu.}
}
%
\maketitle
%

\section*{Introducci\'on}

En astronom\'ia a diferencia de la f\'isica no se pueden realizar experimentos ya
que solo hay un universo observable. Por lo tanto de este se tiene que obtener 
toda la informaci\'on posible por medio de observaciones y con estas poder entender 
los fen\'onemos f\'isicos presentes en el universo.

A partir de estas observaciones, en part\'icular de la espectroscop\'ia se obtiene informaci\'on,
sobre la abundancia de elementos quimicos, las velocidades radiales, tasas de formacion estelar, 
masas de estrellas,movimiento propios etc.

Estas observaciones son de la radiaci\'on proveniente del universo y se hacen
 en diferentes longitudes de onda del espectro electromagnetico. Estas longitudes 
 de onda se dividen en: Microondas, Rayos X, Infrarojo, Visible, Radio, UV. La observaci\'on 
 en estas frecuencias depende en gran medida de las ventanas presentes en la atmosfera terrestre.
Las principales ventanas se encuentran en el rango visible y en ondas de radio por lo que muchos telescopios terrestres observan en estas longitudes de onda. 

El objetivo de este trabajo es familiarizarse con las tecnicas observacionales en astronom\'ia, en particular
con el uso de espectros astron\'omicos, para esto se pretende observar una estrella binaria $\varepsilon$ CRA 
y encontrar su periodo orbital a partir de la medici\'on de su esp\'ectro. El estudio por medio de espectros es muy utilizado en astronomia, estudiar estrellas binarias es de gran importancia ya que se estima que el $50\%$ de estrellas en la via l\'actea son binarias.  

Por otro lado conociendo la orbita de estrellas es posible reconstruir el potencial gravitacional 
del sistema, lo cual es de bastante utilidad ya que muchas veces el potencial gravitacional no es conocido para sistemas complejos (Via l\'actea). reconstruir el potencial gravitacional del sistema binario se deja como 
complemento de este trabajo.\\
\\

\section{Marco Te\'orico}

\subsection{Espectrograf\'ia}

La espectrograf\'ia es una t\'ecninca en la cual la luz se descompone en las diferentes
longitudes de onda. A partir de la intensidad de las diferentes lineas de emisi\'on/absorci\'on
se pueden encontrar cantidades f\'isicas, tales 
como la composici\'on quimica, temperatura superficial, la masa, tasas de formaci\'on estelar
 y si hay presencia de medio interestelar se puede hallar la cinem\'atica del gas [2].
 
Conociendo los espectros estelares es posible reconstruir sint\'eticamente espectros de galaxias
y asi saber las poblaciones estelares presentes en cada galaxia y si se hace esto para galaxias
con diferentes edades es posible ver como evolucionan las poblaciones estelares en las galaxias 
con el tiempo.

La espectrografia es la tecnica mediante la cual se puede obtener mas informaci\'on de la radiacion 
proveniente de los diferentes objetos celestes.

\subsection{Clasificaci\'on espectral de las estrellas}

Esta clasificaci\'on se denomina clasificaci\'on espectral de Harvard en esta las estrellas se clasifican seg\'un su temperatura as\'i:[2]\\

O-B-A-F-G-K-M-L-T\\

\begin{itemize}

\item O son estrellas de azules (calientes) de temperatura superficial entre 20000K y 35000K.\\
\item B son estrellas azules-blancas de temperatura superficial de 15000K.\\
\item A son estrellas blancas de temperatura superficial de 9000K.\\
\item F son estrellas blancas-amarillas con temperatura superficial de 7000K.\\
\item G son estrellas amarillas como nuestro sol con temperatura superficial de 5500K.\\
\item K son estrellas naranjas-amarillas con una temperatura superficial de 4000K.\\
\item M son estrellas rojas de temperatura superficial de 3000K.\\
\item L son estrellas marronas con temperatura superficial de 2000K.\\
\item T son enanas marrones con temperatura superfical de 1000K.\\
\end{itemize}


\subsection{Binarias espectrosc\'opicas}

Las estrellas binarias espectrosc\'opicas solo se pueden detectar mediante sus espectros, estos espectros muestran dos veces las lineas de absorci\'on o emisi\'on una con corrimiento hacia 
el rojo y la otra al azul debido al movimiento orbital de las estrellas, donde la m\'axima separaci\'on sera cuando una estrella se aleja de la
linea de vision y la otra se acerca, el periodo de estas separaciones correspondera al periodo
orbital de la binaria.

Para encontrar la velocidad relativa tenemos que:
\begin{equation}
\dfrac{v}{c} \simeq z = \dfrac{\lambda_{o} - \lambda_{e}}{ \lambda_{e}}
\label{v}
\end{equation} 

\section{Seleci\'on de la binaria a observar}

Para la seleci\'on de la estrella binaria a observar se tuvieron en cuenta 
diferentes caracteristicas tales como:

\begin{itemize}
\item Visibilidad en nuestra ubicaci\'on, (AR $> 16$h, Declinacion $(-40^{0}, 80^{0})$)
\item Magnitud aparante menor a 5, (M $<5$).
\item La duraci\'on del periodo menor a 2 meses.
\item Estrellas calientes para obtener mas lineas de emision y as\'i poder
encontrar la oribta con mayor exactitud.
\end{itemize}

Despues de tener en cuenta los parametros en el cat\'alogo de estrellas binarias[3] encontramos los siguientes candidatos:\\

\begin{tabular}{c c c c c c}
\hline
Nombre & RA & DEC & Periodo (D\'ias) & Tipo espectral & Magnitud\\
\hline
$\varepsilon$ $\mathrm{CRA}$ & 18h59m39s & $-37^{0}05' 17''$ & 0.59 & F0V & 4.8\\
\hline
$\mu 1$ $\mathrm{Sco}$ & 16h52m48s & $-38^{0}04'11''$& 1.44 & B1.5V& 3.00\\
\hline
\\
\end{tabular}
\\
Entre estas dos estrellas se seleccion\'o Epsilon de la Coronae Australis (${\varepsilon}$ CRA) ubicada en la 
constelaci\'on de la Coronae Australis Fig. \ref{la}, ya que su per\'iodo es el mas corto. 


\begin{figure}
\centering
\includegraphics[scale=0.4]{CRA.png}
\caption{Corona Australis [5]\label{la}}
\end{figure}periodoSpica.png


\section{Observaciones}

Todas las observaciones se han llevado acabo en el observatorio astron\'omico de la 
Universidad de los Andes. A continuaci\'on se describen la instrumentacion utlizada
as\'i como los protocolos de observaci\'on utilizados.

El montaje experimental que se utilizo se muestra en la Fig. \ref{montaje} en el cual se ve el acople
de las fibras \'opticas al telescopio, por medio de estas fibras las luz es llevada hasta
el espectr\'ografo.

\subsection{Instrumentaci\'on}

\subsubsection{Telescopio}

Se utiliz\'o un telescopio marca Meade LX200 Schimdt-Cassegrain Fig.\ref{la} de $40 cm$ de apertura y una distancia focal de $4m$.

\begin{figure}señalan
\centering
\includegraphics[scale=0.4]{SCT2.jpg}
\caption{Camino de luz en un telescopio Schmidt-Cassegrain [6] \label{la}}
\end{figure}


\subsubsection{Espectrografo}

El espectr\'ografo que se utiliz\'o Fig.\ref{espectrografo} es un espectrografo de alta resolucion en el cual la luz del telescopio llega por medio de una fibra \'optica, luego esta luz es descompuesta por una rendija de difracci\'on y finalmente la radiaci\'on es recolectada en una CCD.


\begin{figure}
\centering
\includegraphics[scale=0.2]{espectroscopio.jpg}
\includegraphics[scale=0.2]{montaje.jpg}
\caption{Espectrografo (Izquierda), Montaje experimental (Derecha)\label{espectrografo}}
\end{figure}



\subsubsection{Software}

La reducci\'on de datos se llevo acabo con el software ISIS [4]
el cual usa como referencia los esp\'ectros de las lamparas de calibracion (de torio y tungsteno)
para obtener los perfiles de los espectros tomados de las estrellas.

\subsection{Protocolo de Observacion}

Todas las observaciones se han llevado acabo en el observatorio de astronomico
de la universidad de los andes. Los datos ac\'a presentados se tomaron las noches
del 10, 13, 16 de septiembre y el 15 de Octubre 2013. En un intervalo de tiempo aproximadamente de 5 horas aproximadamente. 

El protocolo de observacion que se siguio fue el siguiente:

\begin{itemize}
\item Preparar el montaje, conectar el espectroscopio al telescopio haciendo de las 
fibras opticas.
\item Tomar espectros de las lamparas de calibraci\'on 
\item Posicionar la fibra optica en el foco del telescopio
\item Encontrar la estrella binaria $\varepsilon CRA$ y enfocarla en la fibra optica
\item Tomar los espectros, entre 5 y 15 min cada uno.
\end{itemize}

\section{Discusion y resultados}

De los espectros observados para $\varepsilon CRA$ ninguno presenta 
lineas dobles, sin estas lineas dobles no es posible obtener informaci\'on 
relevante sobra la orbita de la binaria. Por lo tanto con estos esp\'ectros
calcularemos la velocidad radial del sistema, y se usaran los esp\'ectros
de Spica (otra binaria) previamnete tomados en el mismo telescopio con el fin de hacer el analisis de la orbita de Spica.

\subsection{Velocidad radial de $\varepsilon CRA$}

En la tabla\ref{spectra} se encuentra un resumen de los espectros que se usaron para encontrar la velocidad radial.\\

\begin{table}
\begin{center}
\begin{tabular}{|c| c |c|}
\hline
Nombre & Fecha & Hora\\
\hline
$Sept10EpscraA$ & Septiembre 10 & $14.254$\\
\hline
$Sept10EpscraB$ & Septiembre 10 & $14.254$\\
\hline
$Sept10EpscraC$ & Septiembre 10 & $14.254$\\
\hline
$Sept13EpscraA$ & Septiembre 13 & $13.996$\\
\hline
$Sept13EpscraB$ & Septiembre 13 & $14.039$\\
\hline
$Sept13EpscraC$ & Septiembre 13 & $14.062$\\
\hline
$Oct15A$ & octubre 15 & $16.007$\\
\hline
$Oct15B$ & octubre 15 & $16.016$\\
\hline
$Oct15C$ & octubre 15 & $16.032$\\
\hline
$Oct15D$ & octubre 15 & $16.045$\\
\hline
$Oct15E$& octubre 15 & $16.057$\\
\hline
$Oct15F$ & octubre 15 & $16.070$\\
\hline
\end{tabular}
\caption{Espectros utilizados\label{spectra}}
\end{center}
\end{table}


Para hallar la velocidad radial primero identificamos las lineas que tengan un mejor perfil (en su mayoria son las de FeI y FeII) es decir que no sean tan anchas y sobresalgan del continuo.
En la Fig.\ref{espectra} se muestra las lineas que se escogieron para el $Oct15F$. La longitud de onda observada
se escoge respecto el centro de la linea i.e el punto en el cual la intensidad corresponde al FWHM. \\

Las barras de error corresponden al error a la hora de encontrar la longitud observada, as\'i el error en logntiud de onda aparentemente no sea tan significativo $\sim 0.1 \AA$ en velocidades si es considerable $\sim 7 kms^{-1}$. \\ 

Tambien hay presencia de lineas mezcladas, es decir en una aparante linea pueden haber dos o m\'as lineas de distintos 
elementos. Esto hace que el centro de la linea se vea modificiado lo cual cambiaria la longitud de onda observada, 
un m\'etodo para evitar esto es ajustar funciones Gaussianas a cada una de las lineas y as\'i separarlas y saber con exactitud el centro de la linea.\\ 

Las observaciones fueron corregidas por la velocidad propia de osbervatorio de la Universidad respecto al Sol que es el sistema de referencia para las verlocidades reportadas por Bilir et al (2005) 

\begin{figure}
\includegraphics[scale=0.3]{spectra.png}
\caption{Espectros de $\varepsilon CRA$ tomados el 15/oct/2013
las lineas punteadas corresponden a la longitud de onda observada, las 
lineas solidas corresponden a la longitud de onda emitida esta ulitma se tomo de Bilir et al(2005)\label{espectra}}
\end{figure}

Haciendo uso de \eqref{v} se encontro la velocidad para cada linea Fig\ref{vradial} cuyo valor promedio fue de $55.285 \pm 6.59 Km s^{-1}$ el error porcentual respecto al valor reportado por \ref{brimel} $57.9 \pm 1.2 Kms^{-1}$ fue de $4.51\%$. \\

\begin{figure}
\includegraphics[scale=0.5]{radialvelocity.png}
\caption{Velocidad Radial de cada una de las lin\'eas en funci\'on de la longitud de onda. La linea roja representa el valor reportado por S.Bilir et al (2005), la azul punteada el valor promedio encontrado con [nombre dle espectro] de $v=55.285 \pm 6.59 Kms^{-1}$\label{vradial}}
\end{figure}

Con el fin de buscar alg\'un indicio del la orbita en el esp\'ectro de $\varepsilon CRA$ buscamos efectos tales como 
el ensanchamiento de la linea o alguna relacion velocidad radial/fase. Donde la fase representa el momemnto en el cual el 
sistema binario esta.\\

Para esto se utilizaron todos los espectros tomados en Octubre 15. En la Fig.\ref{AllSpectra} se muestra como el centro de la linea se ve ligeremante afectado por la diferencia de tiempo en la que son tomados los espectros.\\

\begin{figure}
\includegraphics[scale=0.5]{epscraEspectra.png}
\caption{Velocidad radial para todos los espectros de Octubre 15. \label{AllSpectra}}
\end{figure}

Para hallar la fase de $\varepsilon CRA$ dividimos el la hora a la que fue la observacion por el periodo te\'orico de 14h[XXX] y restamos la parte entera.\\

En las Fig.\ref{FWHM} se ve una dependencia con la fase, principalmente al aumentar la fase el FWHM tiene a incrementarse esto da muestra de que las binarias estan llegando a su maxima separacion respecto a linea de observacion.\\

\begin{figure}
\includegraphics[scale=0.5]{anchovsfase.png}
\caption{FWHM en funcion de la fase 15. \label{FWHM}}
\end{figure}


Todos los datos tomados estan disponible en:

$ https://github.com/jngaravitoc/EpsCra/tree/master/data.$

El desarrollo para encontrar las velocidades radiales esta 
disponible en:

$http://nbviewer.ipython.org/urls/raw.github.com/jngaravitoc/EpsCra/master/Spectra_analysis.ipynb
$
\subsection{Periodo Orbital de Spica}

Con el fin de estudiar la orbita de una estrella binaria, se utilizaron los espcetros tomados en Junio y Julio en el observatorio de la Universidad de los Andes de la binaria Spica.\\

En estos espectros Fig.\ref{dp} se observa claramente la presencia de dos picos de HeI esto debido a que cada estrella emite su propio espectro. Tambien se observa como a medida que pasa el tiempo los dos picos presentan movimineto relativos entre ellos, en part\'icular se acercan.\\

\begin{figure}
\includegraphics[scale=0.5]{doblepeak.png}
\caption{Doble pico de HeI presentado en el espectro de Spica \label{dp}}
\end{figure}

Ahora bien para obtener informaci\'on del periodo se grafica la velocidad en funci\'on del tiempo y se debe ver una oscilacion en las dos velocidades. Pero para esto es necesario tener suficientes medidas al menos de 4 dias consecutivos, y las medidas realizadas son de dos dias con una seperacion entre ellos de 28 d\'ias, por lo cual no es posible obtener una resultado con nuestros datos. En a Fig\ref{pS} las lineas punteadas muestran el periodo teorico de Spica mientas que las solidas muestran nuestros datos, como se puede ver aun no es posible encontrar la orbita de Spica.

\begin{figure}
\includegraphics[scale=0.5]{periodoSpica.png}
\caption{Periodo de Spica, lineas no continuas representan el periodo teorico, lineas solidos representan los datos observados. El color distingue a cada una de las estrellas \label{pS}}
\end{figure}


\section{Conclusiones}

\begin{itemize}
\item Se encontro la velocidad radial de $\varepsilon CRA$ con un error relativo del $4\%$, lo cual demuestra que es posible obtener velocidades radiales de estrellas con una buena precisi\'on.

\item Se observo un ensanchamiento en la misma linea de diferetes espectros, esto se debe a la fase en que se encuentra $\varepsilon CRA$.

\item No fue posible resolver el sistema binario de $\varepsilon CRA$ ya que estas estrellas es\'an muy cerca entre ellas. 

\item Es posible obtener el periodo de la binaria Spica pero es necesario mas dias de observaci\'on lo cual no siempre es posbile dadas las condiciones clim\'aticas de Bogot\'a.  
\end{itemize}

\section{Agradecimientos}

JNG-C agradece al Prof. Benjamin Oostra por su infantable asesoria y por su tiempo dedicado a este trabajo. JNG-C AGRADECE tambien a Juan Camilo Buitrago quien participo de manera activa en todas las noches de observaci\'on.  

\section{Referencias}

1. $http://ned.ipac.caltech.edu$  \\
2. Karttunen, Fundamental astronomy 5th edition.\\
3. Alan H. Batten, J. Murray Fletcher and D. G. MacCarthy, http://ad.usno.navy.mil/wds/dsl/SB8/sb8.html\\
4. $http://www.astrosurf.com/buil/isis/isis_en.htm$\\
5. http://www.iau.org/static/public/constellations/gif/CRA.gif \\
6. http://en.wikipedia.org/wiki/File:Schmidt-Cassegrain-Telescope.png \\
7. http://simbad.u-strasbg.fr/simbad/sim-id?Ident=%402353328&Name=V*%20eps%20CrA&submit=submit
8. Brimil et al (2005), MNRAS
\end{document}
